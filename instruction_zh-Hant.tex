% TODO: 目前内容不足以凑满两页(正反 A4 纸),所以有些内容注释掉了,之后完善

\documentclass{article}
\setlength\parindent{0pt}
\pagenumbering{gobble}

\usepackage{geometry}
\geometry{papersize={210mm,297mm}}
\geometry{left=9mm,right=9mm,top=9mm,bottom=9mm}

\usepackage{fontspec}
\setmainfont{DejaVu Sans}[Scale=0.9]

\usepackage{xeCJK}
\CJKfamily{zhhei} % texlive 不知道有啥自带的能用的正体字形,先这样吧。

\usepackage{longfbox}
\usepackage{epstopdf}
\usepackage{float}
\usepackage{xpatch}
\usepackage{tipa}
\usepackage[export]{adjustbox}
\usepackage{enumitem}
\usepackage{mdframed}

\usepackage{multirow}
\usepackage{tabularx}
\renewcommand\tabularxcolumn[1]{m{#1}} % for vertical centering text in X column

\usepackage{multicol}
\usepackage[most]{tcolorbox}
\usepackage{amssymb}
\setlength{\columnsep}{3mm}

\usepackage[explicit,compact]{titlesec}
\titleformat{\section}{\normalfont\bfseries}{}{0pt}{【#1】}

% 啊!万能的 StackExchange!
% https://tex.stackexchange.com/questions/475466/latex-three-column-layout-merging-two-of-them-at-the-begining
%
\newlength{\abstractwidth}
\newlength{\columnshrink}
\newsavebox{\twocolinsert}
%
\makeatletter
\newlength{\resized@col}
\newcounter{column@count}
%
\xpatchcmd{\multi@column@out}{
	\process@cols\mult@firstbox{%
		\setbox\count@
		\vsplit\@cclv to\dimen@
		\set@keptmarks
		\setbox\count@
		\vbox to\dimen@
		{\unvbox\count@ \ifshr@nking\vfilmaxdepth\fi}%
	}%
}{
	\process@cols\mult@firstbox{%
		\global\advance\c@column@count\@ne
		\resized@col\dimen@%
		\ifnum\c@column@count=\tw@
				\advance\resized@col-\columnshrink
		\fi%
		\setbox\count@
		\vsplit\@cclv to\resized@col
		\set@keptmarks
		\setbox\count@
		\vbox to\dimen@{
			\ifnum
				\c@column@count=\tw@ \vspace*{\columnshrink}
			\fi
			\unvbox\count@
			\ifshr@nking\vfilmaxdepth\fi
		}%
	}%
}{\typeout{Success}}{\typeout{Failure}}
\makeatother

% for designing header
\newsavebox\mysavebox
\newenvironment{imgminipage}[2][]{%
   \def\imgcmd{\includegraphics[width=\wd\mysavebox, height=\dimexpr\ht\mysavebox+\dp\mysavebox\relax, #1]{#2}}%
   \begin{lrbox}{\mysavebox}%
   \begin{minipage}%
}{%
   \end{minipage}
   \end{lrbox}%
   \sbox\mysavebox{\setlength{\fboxrule}{0pt}\fbox{\usebox\mysavebox}}%
   \mbox{\rlap{\raisebox{-\dp\mysavebox}{\imgcmd}}\usebox\mysavebox}%
}

\renewcommand{\labelitemi}{$\blacktriangleright$}

\tcbset{
    frame code={}
    center title,
    left=0pt,
    right=0pt,
    top=6pt,
    bottom=0pt,
    colback=gray!40,
    colframe=white,
    enlarge left by=0mm,
    boxsep=0pt,
    arc=0pt,outer arc=0pt,
}

\begin{document}
\begin{multicols*}{3}

	\setlength{\abstractwidth}{2\linewidth}
	\addtolength{\abstractwidth}{\columnsep}
	\savebox{\twocolinsert}{\begin{minipage}{\abstractwidth}
		\noindent 核准日期:2025年08月09日
		\newline 修改日期:2019年07月06日,2020年10月31日,2021年07月04日,
		\newline 2021年08月14日,2023年06月10日,2025年08月09日
		\newline

		% \begin{mdframed}[leftline=false, rightline=false, innertopmargin=0pt, innerbottommargin=0pt, innerrightmargin=0pt, innerleftmargin=2em]
		% 	\includegraphics[width=0.15\abstractwidth, valign=m]{assets/debian-text.eps}
		% 	\hfill
		% 	\begin{imgminipage}{assets/header-background.eps}[t]{0.7\abstractwidth}
		% 		\Large \textbf{盒装安装媒介说明书}

		% 		\normalsize 请仔细阅读说明书并在管理员指导下使用
		% 	\end{imgminipage}
		% \end{mdframed}

		\includegraphics[width=\abstractwidth]{assets/header.eps}


		\begin{mdframed}[hidealllines=true, innerbottommargin=.5em, innertopmargin=0pt]
			\sffamily

			{\centering 警告 \par}

			無論是否與其它作業系統合用,安裝 Debian 均存在遺失磁碟上所有內容的風險。 (參見【不良反應】)

			某些與多媒體相關的軟體,特別是允許回放和提供音訊、影片操作或類似功能的軟體,不被 Debian 包含,這是因為它在世界上的某些地區被認為是非法的。本說明中的資訊和意見無意構成法律建議,法律建議可透過諮詢律師來取得。

			一些硬體製造商拒絕告訴我們如何為他們的硬體編寫驅動程式。另一些則要求簽署不公開的協定才能接觸相關文件,以阻止我們發布驅動程式原始碼這項自由軟體的核心內容。由於我們未被授權使用這些文檔,因此它們無法在 Linux 下運作。
		\end{mdframed}
	\end{minipage}}
	\setlength{\columnshrink}{\ht\twocolinsert}
	\addtolength{\columnshrink}{\dp\twocolinsert}
	\noindent\usebox{\twocolinsert}


	\begin{tcolorbox}
	\section*{發行版名稱}
	\end{tcolorbox}
	\begin{tabularx}{\linewidth}{@{}ll@{}}
		通用名稱: & Debian \\
		正式名稱: & Debian GNU/Linux \\
		英文讀音: & \textipa{["dEbi@n]} \\
	\end{tabularx}

	\medskip


	\begin{tcolorbox}
	\section*{內容}
	\end{tcolorbox}

	完全由自由軟體組成的類 UNIX 作業系統,其包含的多數軟體使用 GNU 通用公共授權協議授權,並由 Debian 計畫的參與者組成團隊對其進行打包、開發與維護。

	% 内核版本:4.19.0

	% 版本号:“buster”

	\medskip


	\begin{tcolorbox}
	\section*{性質}
	\end{tcolorbox}

	本系統為採用 GNU/Linux 內核的作業系統,安裝後可由 UEFI 或 Legacy 引導方式啟動。

	\medskip


	\begin{tcolorbox}
	\section*{適用平台}
	\end{tcolorbox}

	\begin{itemize}  % 把信创原神放前面,其他没人用的放后面
		\item 可用於 riscv64、arm64、amd64、armhf、ppc64el、s390x 等架構的電腦、伺服器和嵌入型裝置。
		\item 本包裝盒中的安裝媒體適用於何平台以實際為準。
	\end{itemize}


	\begin{tcolorbox}
	\section*{規格}
	\end{tcolorbox}

	1 枚 安裝媒體

	\medskip

	\begin{tcolorbox}
	\section*{用法}
	\end{tcolorbox}

	使用 USB 裝置引導。

	啟動方式依硬體調整,一般使用 UEFI。

	根據硬體效能和個人需要,調整安裝方式:一般而言,使用圖形化介面進行安裝;否則選擇基於 ncurses 命令列的安裝。

	安裝基本系統並設定軟體套件後,即可視情況選擇桌面環境和桌面管理員。

	\medskip

	\begin{tcolorbox}
	\section*{不良反應}
	\end{tcolorbox}

	Debian 有一個對使用者和開發者所提交的軟體缺陷報告進行歸檔管理的缺陷追蹤系統,英文縮寫為 BTS。每個軟體缺陷報告都被授予一個編號並且被長期跟踪,直到它被標記為已修復。

	可以在 https://www.debian.org/Bugs/ 獲取該檔案的複本。

	\medskip


	\begin{tcolorbox}
	\section*{注意事項}
	\end{tcolorbox}
	\begin{itemize}[leftmargin=*]

		\item 滿足運行的最低要求

		Pentium 4、1GHz 的處理器是桌上型系統的最低建議配置,下表是記憶體和硬碟的需求。

		{\small\begin{tabularx}{\linewidth}{|X|X|X|X|}
			\hline
			類別 & RAM\newline (最低) & RAM\newline (推薦) & 硬碟 \\
			\hline
			無桌面 & 512MB & 1GB & 4GB \\
			\hline
			有桌面 & 1GB & 2GB & 10GB \\
			\hline
		\end{tabularx}}

		基於您的需求,也許可以使用低於上表所列的設定完成系統安裝。但是多數用戶在無視這些建議的情況下會安裝失敗。

		\item 需要韌體的裝置

		除了需要裝置驅動程式,有些硬體還要在使用前載入韌體(firmware)或微程式(microcode)。

	\end{itemize}


	\begin{tcolorbox}
	\section*{禁忌}
	\end{tcolorbox}

	在使用中若出現或即將出現下列任何一種情況,請立即停止使用,並準備好系統恢復。

	\begin{itemize}[leftmargin=*]
		\setlength{\itemsep}{0pt}
		\setlength{\parskip}{0pt}
		\setlength{\parsep}{0pt}

		\item 以根權限在根目錄下執行遞迴刪除
		\item 未確認裝置代號即使用 dd 指令
		\item 未確認裝置代號即進行格式化
		\item 未確認操作即使用指令重新導向
		\item 未經確認就運行來自網路的指令集
		\item 長期在散熱不良的裝置上高負荷使用
	\end{itemize}


	% \begin{tcolorbox}
	% \section*{无障碍安装}
	% \end{tcolorbox}

	% Debian GNU/Linux 安装介质不用于视力或运动障碍人士。

	% \medskip

	% \begin{tcolorbox}
	% \section*{新手安装}
	% \end{tcolorbox}

	% Debian GNU/Linux 应谨慎用于新手安装,需要在管理员指导下进行安装,并且需要进行密切的系统监测,一旦出现系统完整度的恶化,应考虑停止使用 Debian GNU/Linux。

	% \medskip


	% \begin{tcolorbox}
	% \section*{版本迭代}
	% \end{tcolorbox}


	\begin{tcolorbox}
	\section*{系統相互作用}
	\end{tcolorbox}
	\begin{itemize}[leftmargin=*]
		\setlength{\parindent}{0pt}

		\item 与 Windows 的相互作用

		當您有雙重引導時,若另一個作業系統與 Windows 存取相同的檔案系統,這就有可能會導致問題和資料遺失。在這種情況下,檔案系統的真實狀態可能與 Windows 認為在「啟動」之後的情況不同,並且可能在進一步寫入檔案系統時導致檔案系統損毀。因此,在雙重開機設定中,為了避免檔案系統損毀,有必要在 Windows 中停用「快速啟動」功能。


		在罕見情況中已觀察到,在使用 Windows 進行系統更新時,可能會出現重新啟動後 GRUB 引導被破壞從而導致 Debian 無法啟動的情況。同時,若在安裝過程中將引導資訊寫入 Windows 所在的實體磁碟 MBR 內,將導致後者無法正常啟動。

		\item 與其他 Linux 發行版本的相互作用

		尚不明確。

	\end{itemize}


	\begin{tcolorbox}
	\section*{儲存}
	\end{tcolorbox}

	-40℃\textasciitilde +70℃

	妥善儲存所有安裝媒體,避免非技術人員接觸。

	\medskip


	\begin{tcolorbox}
	\section*{包裝}
	\end{tcolorbox}

	裝有 Debian GNU/Linux 安裝媒體的、支援 USB 3.0/2.0 協定的大容量存儲裝置。

	1 枚/盒

	\medskip


	\begin{tcolorbox}
	\section*{有效期限}
	\end{tcolorbox}

	完整支援:36 个月

	長期支援:60 个月

	\medskip


	\begin{tcolorbox}
	\section*{執行標準}
	\end{tcolorbox}
	\begin{tabularx}{\linewidth}{@{}ll@{}}
		\multirow{4}{*}{}{開放原始碼許可:} & GNU GPL(主要)\\
		~ & GNU LGPL \\
		~ & BSD \\
		~ & 以及其他授權協定 \\
	\end{tabularx}

	\medskip


	\begin{tcolorbox}
	\section*{批准文號}
	\end{tcolorbox}

	說明書使用 CC-BY-SA 3.0 協定授權。
	\medskip


% 	\begin{tcolorbox}
% 	\section*{生产单位}
% 	\end{tcolorbox}
%
% 	Debian 计划
%
% 	\medskip


	\begin{tcolorbox}
	\section*{說明書}
	\end{tcolorbox}
	\begin{tabularx}{\linewidth}{@{}ll@{}}
		\multirow{2}{*}{}{編審:} & @YukariChiba\\
		~ & @moesoha \\
		圖形:& @YJBeetle\\
		繁體中文版編寫:& @pea-snow
		GitHub: & moesoha/debian-media-box\\
	\end{tabularx}

	\medskip


	\vfill
	\begin{flushright}
		Debian GNU/Linux - 13.0\linebreak 20250809
		\linebreak
		\newline
		\begin{minipage}{0,5\textwidth}
			\centering
			$\vcenter{\hbox{\includegraphics[height=10mm]{assets/debian-logo.eps}}}$
			$\vcenter{\hbox{\includegraphics[height=5mm]{assets/debian-text.eps}}}$
		\end{minipage}
	\end{flushright}

\end{multicols*}
\end{document}
